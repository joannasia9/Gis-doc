\newpage
\fancyhead[C]{ZAŁĄCZNIKI}
\fancyhead[L]{}
\fancyhead[R]{}
\begin{center}
	\textbf{Załącznik 1.} Kod źródłowy algorytmu Prima
\end{center}

\begin{lstlisting}
	public class PrimAlgorithm{
	/**
	* Element losowosci
	*/
	Random rand;
	/**
	* Lista odwiedzonych wierzcholkow
	*/
	private ArrayList<Integer> visitedVertices;
	
	/**
	* Lista wszystkich wierzcholkow
	*/
	private ArrayList<Edge> adjacencyEdgesList;
	
	
	/**
	* Konstruktor
	* Odpowiada za wykonanie poszczegolnych krokow algorytmu
	*
	* @param  graph  macierz sasiedztwa reprezentujaca graf
	*
	*/
	public PrimAlgorithm(final int[][] graph){
	this.adjacencyEdgesList = new ArrayList<>();
	
	this.visitedVertices = new ArrayList<>();
	
	/**
	* Zbior krawedzi
	* Reprezentuje minimalne drzewo rozpinajace
	*/
	Set<Edge> minSpanningTree = new HashSet<>();
	
	/**
	* Zmienna pomocnicza
	* Lista krawedzi
	* Reprezentuje krawedzie incydentne do odwiedzonych wierzcholkow grafu
	*/
	ArrayList<Edge> adjacencyEdges;
	
	/**
	* Zmienna pomocnicza
	* Element odpowiedzialny za losowy wybor wierzcholka poczatkowego
	*/
	this.rand = new Random();
	
	/**
	* Losowy wybor wierzcholka poczatkowego
	* Z zakresu od 0 do n-1,
	* gdzie n-liczba wierzcholkow grafu
	*/
	int startV = rand.nextInt(graph.length-1);
	
	/**
	* Oznaczenie wierzcholka startowego jako odwiedzony
	* Dodanie wierzcholka startV do listy odwiedzonych wierzcholkow
	*/
	setVertexVisited(startV,visitedVertices);
	
	
	/**
	* Pobranie do zmiennej adjacencyEdges listy krawedzi incydentnych do startV
	*/
	adjacencyEdges = getAllAdjacencyEdges(startV,graph, adjacencyEdgesList);
	
	/**
	* Petla warunkowa while()
	* Wykonuje sie dopoki lista odwiedzonych wierzcholkow visitedVertices
	* nie zawiera wszystkich wierzcholkow grafu
	*/
	while(visitedVertices.size()!= graph.length) {
	/**
	* Zmienna reprezentujaca krawedz z listy krawedzi incydentnych
	* do punktow odwiedzonych adjacencyEdges, ktora ma minimalna wage
	*/
	Edge minEdge = adjacencyEdges.get(0);
	/**
	* Instrukcja warunkowa if()
	* Pod warunkiem, ze lista odwiedzonych wierzcholkow visitedVertices
	* nie zawiera wierzcholka koncowego krawedzi minEdge:
	*
	* Do MST zostaje dodana krawedz minEdge
	* Wierzcholek startV zmienia wartosc na wierzcholek koncowy minEdge
	* Krawedz minEdge zostaje usunieta z listy krawedzi incydentnych
	* do rozpatrzenia
	* Lista krawedzi incydentnych adjacencyEdges zostaje zaktualizowana 
	* o krawedzie
	* incydentne do wierzcholka startV
	*
	* W przeciwnym wypadku:
	* Krawedz minEdge zostaje usunieta z listy krawedzi do rozpatrzenia
	*/
	if(!visitedVertices.contains(minEdge.getEnd())){
	minSpanningTree.add(minEdge);
	setVertexVisited(minEdge.getEnd(), visitedVertices);
	startV = minEdge.getEnd();
	adjacencyEdges.remove(minEdge);
	adjacencyEdges = getAllAdjacencyEdges(startV,graph,adjacencyEdgesList);
	} else adjacencyEdges.remove(minEdge); 	}
	/**
	* Wypisanie otrzymanego MST w konsoli
	*/
	printMST(minSpanningTree); }
	/**
	* @param vertex
	* @param list
	* Oznacza wierzcholek jako odwiedzony poprzez dodanie wierzcholka vertex
	* do listy wierzcholkow odwiedzonych list
	*/
	private void setVertexVisited(int vertex, ArrayList<Integer> list){
	/**
	* Instrukcja warunkowa if()
	* Blok instrukcji zostaje wykonany, gdy lista wierzcholkow odwiedzonych 
	* nie zawiera jeszcze wierzcholka vertex
	*/
	if(!list.contains(vertex))
	list.add(vertex);
	}
	
	/**
	*
	* @param vertex
	* @param graph
	* @return
	*
	* Dodaje do listy adjacencyEdgesList krawedzie incydentne 
	* do wierzcholka vertex, ktore nie tworza cyklu
	*/
	private ArrayList<Edge> getAllAdjacencyEdges(int vertex, 
										int[][] graph, ArrayList<Edge> adjacencyEdgesList){
	
	/**
	* Instrukcja iteracyjna for()
	* Blok instrukcji jest wykonywany dla kazdego wierzcholka grafu
	*/
	for(int i=0; i<graph.length; i++){
	/**
	* Instrukcja warunkowa if()
	* Operacja dodania krawedzi incydentnej do wierzcholka vertex 
	* zostaje wykonana pod warunkiem, ze:
	* Krawedz (vertex-i) istnieje, czyli jej waga jest rozna od 0 oraz
	* Lista odwiedzonych wierzcholkow visitedVertices 
	* nie zawiera jeszcze i-tego wierzcholka
	*/
	if(graph[vertex][i]!= 0 && !visitedVertices.contains(i))
			adjacencyEdgesList.add(new Edge(vertex,i,graph[vertex][i]));
	}
	
	/**
	* Sortowanie otrzymanej listy krawedzi wzgledem wag
	*/
	Collections.sort(adjacencyEdgesList);
	
	return adjacencyEdgesList;
	}
	
	/**
	*
	* @param set
	*
	* Wyswietla w konsoli liste krawedzi otrzymanego drzewa
	* wraz z suma wag krawedzi
	*/
	private void printMST(Set<Edge> set){
	int sum = 0;
	for(Edge item : set){
	System.out.println(item);
	sum+=item.getWeight();
	}
	System.out.println("SUMA WAG WYNOSI: " + sum);
	}
\end{lstlisting}


\begin{center}
	\textbf{Załącznik 1.a.} Kod źródłowy klasy Edge
\end{center}

\begin{lstlisting}
	/**
	* Klasa Edge, ktorej obiekty reprezentuja krawedzie grafu
	*/
	private class Edge implements Comparable<Edge> {
	/**
	* Wierzcholek poczatkowy krawedzi
	*/
	private int start;
	/**
	* Wierzcholek koncowy krawedzi
	*/
	private int end;
	/**
	* Waga krawedzi (start - end)
	*/
	private int weight;
	
	/**
	*
	* @param s
	* @param e
	* @param w
	*
	* Konstruktor krawedzi
	*/
	Edge(int s, int e, int w){
	setStart(s);
	setEnd(e);
	setWeight(w);
	}
	
	private int getWeight(){
	return weight;
	}
	
	private int getStart() {
	return start;
	}
	
	private void setStart(int start) {
	this.start = start;
	}
	
	private int getEnd() {
	return end;
	}
	
	private void setEnd(int end) {
	this.end = end;
	}
	
	private void setWeight(int weight) {
	this.weight = weight;
	}
	
	/**
	*
	* @param e
	* @return
	* Metoda odpowiedzialna za porownywanie krawedzi
	* Wykorzystywana podczas sortowania krawedzi w liscie
	* @see Collections
	*/
	@Override
	public int compareTo(Edge e) {
	if (getWeight() < e.getWeight()){
	return -1;
	} else if (getWeight() > e.getWeight()) {
	return 1;
	}
	return 0;
	}
	
	/**
	* @return
	*
	* Metoda odpowiedzialna za postac wypisanej w konsoli krawedzi
	*/
	@Override
	public String toString() {
	return String.format("%d--%d\tw:%d ", start, end, weight);
	}
	}
\end{lstlisting}
\newpage
\begin{center}
	\textbf{Załącznik 2.} Kod źródłowy modułu odpowiedzialnego za odczyt macierzy z pliku
\end{center}

\begin{lstlisting}
	import java.io.*;
	
	public class GraphFileExtractor {
	private Graph graph;
	private int[][] adjacencyMatrix;
	
	public void getFileContent(int iterator) {
	String home = System.getProperty("user.home");
	File file = new File(home + File.separator + "Desktop" 
											+ File.separator + "G" + iterator + ".txt");
	
	try {
	BufferedReader br = new BufferedReader(new FileReader(file.getPath()));
	String line = br.readLine();
	initiateMatrix(line);
	
	int i = 0;
	while (line != null) {
	parseLine(i,line);
	line = br.readLine();
	i+=1;
	}
	
	} catch (IOException e) {
	e.printStackTrace();
	}
	}
	
	private void parseLine(int i, String line){
	String delim = "\t";
	String[] tokens = line.split(delim);
	
	for(int j=0; j<tokens.length;j++){
	if(Integer.parseInt(tokens[j])!=0)
	graph.addEdge(new Edge(i,j,Integer.parseInt(tokens[j])));
	adjacencyMatrix[i][j] = Integer.parseInt(tokens[j]);
	}
	
	}
	
	private void initiateMatrix(String line){
	String delim = "\t";
	String[] tokens = line.split(delim);
	graph = new Graph(tokens.length);
	adjacencyMatrix = new int[tokens.length][tokens.length];
	}
	
	public Graph getGraph(){
	return graph;
	}
	
	public int[][] getAdjacencyMatrix(){
	return adjacencyMatrix;
	}
	}
\end{lstlisting}