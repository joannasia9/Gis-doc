\section{Ogólne założenia dotyczące badań algorytmów}
\begin{enumerate}
	\item Algorytmy zostaną wykonane dla \textbf{grafów spójnych, nieskierowanych}
	\item Działający program zostanie przetestowany na przypadkach, dla których znane są poprawne rozwiązania 
	\item Na wejściu każdego z algorytmów dany jest graf w postaci \textbf{macierzy wag} o zadanej liczbie wierzchołków \emph{N}(ang. nodes) oraz \emph{E} krawędzi (ang. edges), gdzie $1 \leqslant N \leqslant 10000$) oraz $1 \leqslant M \leqslant 100000$.
	\begin{center}
		\textbf{Macierz sąsiedztwa}
	\end{center}
Graf reprezentujemy za pomocą macierzy kwadratowej \emph o stopniu \emph{n}, gdzie n oznacza liczbę wierzchołków w grafie. Macierz tą nazywamy macierzą sąsiedztwa (ang. adjacency matrix). Odwzorowuje ona połączenia wierzchołków krawędziami. Wiersze macierzy sąsiedztwa odwzorowują zawsze wierzchołki startowe krawędzi, a kolumny odwzorowują wierzchołki końcowe krawędzi. Komórka \emph{A[i,j]}, która znajduje się w\emph{ i-tym} wierszu i \emph{j-tej } kolumnie odwzorowuje krawędź łączącą wierzchołek startowy vi z wierzchołkiem końcowym \emph{$v_{j}$}. Jeśli \emph{A[i,j]} ma wartość 1, to dana krawędź istnieje. Jeśli \emph{text} ma wartość 0, to wierzchołek \emph{$v_{i}$} nie łączy się krawędzią z wierzchołkiem \emph{$v_{j}$}.
\item Na wyjściu każdego z algorytmów otrzywana zostanie suma wag krawędzi MST
\item W związku z tym, że wejście oraz wyjście obu algorytmów dotyczy tych samych danych, wyniki zostaną porównane
\item Głównym elementem podlegającym porównaniu będzie czas wykonania obu algorytmów
\item \textbf{Złożoność czasowa} -- ze względu na wykorzystane struktury danych, złożoność czasowa wybranych algorytmów powinna wynosić odpowiednio:
\begin{itemize}
	\item Dla algorytmu Prima: \emph{O(E$\cdot$ log N)}
	\item Dla algorytmu Kruskala: \emph{O(E$\cdot$ log N)}
\end{itemize}
\end{enumerate}