%Usuwa numeracje z naglowka. Zapewnia  dodanie do spisu tresci

%Gdy mamy dużą bibliografię to możemy wybierać pozycje,
%które cytujemy
%\nocite{ad-tg-80}

%Dodaje wszystkie pozycje z bibliografii
%\nocite{*}

%Po kazdym dodaniu nowej pozycji bibliograficznej
%z katalogu glownego uruchom: bibtex pracadyp
%\bibliographystyle{pdplain}
%\bibliography{tex/pracadyp}


\setcounter{secnumdepth}{-1}
\begin {thebibliography}{11}
\fancyhead[R]{BIBLIOGRAFIA}
\bibitem{prim} \emph{http://www.algorytm.org/algorytmy-grafowe/algorytm-prima.html}
\bibitem{prim2}\emph{http://algorytmika.wikidot.com/mst}
\bibitem{prim3}\emph{http://eduinf.waw.pl/inf/alg/001\_search/0141.php}
\end {thebibliography}
\listoffigures
\addcontentsline{toc}{chapter}{Spis tabel}
