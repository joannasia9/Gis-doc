%Usuwa numeracje z naglowka. Zapewnia  dodanie do spisu tresci

%Gdy mamy dużą bibliografię to możemy wybierać pozycje,
%które cytujemy
%\nocite{ad-tg-80}

%Dodaje wszystkie pozycje z bibliografii
%\nocite{*}

%Po kazdym dodaniu nowej pozycji bibliograficznej
%z katalogu glownego uruchom: bibtex pracadyp
%\bibliographystyle{pdplain}
%\bibliography{tex/pracadyp}

\listoffigures
\setcounter{secnumdepth}{-1}
\begin {thebibliography}{11}
\fancyhead[R]{BIBLIOGRAFIA}
\bibitem{prim} \emph{http://www.algorytm.org/algorytmy-grafowe/algorytm-prima.html}
\bibitem{prim2}\emph{http://algorytmika.wikidot.com/mst}
\bibitem{gis}Jacek Wojciechowski, Krzysztof Pieńkosz: \emph{Grafy i sieci},  Wydawnictwo Naukowe PWN SA, Warszawa 2013
\bibitem{gis2} \emph{http://www.ams.org/proc/1956-007-01/S0002-9939-1956-0078686-7/}
\end {thebibliography}

